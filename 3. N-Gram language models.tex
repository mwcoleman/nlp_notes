\documentclass{article}
\usepackage{amsmath}

\begin{document}

\section{N-Gram Language Models}
\subsection{Definition}
\paragraph{Applying the markov assumption to text for the purpose of building up a language model (a distribution of probabilities):}
\begin{align}
	P(w_i | w_{<i}) &= P(w_i|w_{i-1})
\end{align}
\paragraph{The above specific case is a 1st order markov model- a 'bigram' language model; each word's probability is conditioned on the preceding word only. We can extend this to an nth order markov model- an n-gram:}
\begin{align}
	P(w_i|w_{i-n+1 < i})
\end{align}
\subsection{Learning the probabilities - MLE}
\begin{align}
	P(w_i|w_{i-n+1 < i}) &= \frac{\# counts(w_{i-n+1 < i}, w_i)}{\# counts(w_{i-n+1 <i})}
\end{align}
\subsection{Evaluating models}
\paragraph{Evaluating the probability of a sequence of text based on the language model could be done a bunch of ways, but if it were simply the joint probability then it would be sensitive to the sequence length.}
\paragraph{So, commonly, a measure called 'perplexity' is used; the inverse joint probability normalised by the length (\# words)}
\begin{align}
	PP(W) &= \sqrt[N]{\frac{1}{\prod_i^N P(w_i|w{i-n+1}<i)}} 
\end{align}




\end{document}



